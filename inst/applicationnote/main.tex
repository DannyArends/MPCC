\PassOptionsToPackage{utf8}{inputenc}
\documentclass{bioinfo}

\copyrightyear{2020} \pubyear{2020}

\access{Advance Access Publication Date: day month 2020}
\appnotes{Application Note}

\begin{document}
\firstpage{1}

\subtitle{Genetic and population analysis}

\title[MPCC]{A Performant Matrix of Pearson$'$s Correlation Coefficient (MPCC) Calculations}
\author[Arends \textit{et~al}.]{
Danny Arends\,$^{\text{\sfb 1, $\dagger$}}$,
Mitch Horton\,$^{\text{\sfb 2, $\dagger$}}$,
Chad Burdyshaw\,$^{\text{\sfb 2, $\dagger$}}$,
Udit Gulati\,$^{\text{\sfb 3}}$,
Christian Fischer\,$^{\text{\sfb 4}}$,
Robert W. Williams\,$^{\text{\sfb 4}}$,
Pjotr Prins\,$^{\text{\sfb 4}}$,
Glenn Brook\,$^{\text{\sfb 2, *}}$}
\address{$^{\text{\sf 1}}$Z{\"u}chtungsbiologie und molekulare
Genetik, Albrecht Daniel Thaer-Institut, Berlin, 10115, Germany \\
$^{\text{\sf 2}}$The Joint Institute for Computational Sciences,
University of Tennessee, Oak
Ridge, TN 37830, USA\\
$^{\text{\sf 3}}$Computer Science Dept. Indian Institute of
Information Technology, Una, Himachal Pradesh, India\\
$^{\text{\sf 4}}$Genetics, Genomics and Informatics, University
of Tennessee Health Science Center, Memphis, TN 38163, USA.}

\corresp{$^\dagger$Contributed equally and should be considered
joined first authors, $^\ast$To whom correspondence should be
addressed.}

\history{Received on XXXXX; revised on XXXXX; accepted on XXXXX}

\editor{Associate Editor: XXXXXXX}

\abstract{\textbf{Motivation:}
The work presented is motivated by GeneNetwork.org, which performs a
matrix of Pearson$'$s Correlation Coefficient (PCC) calculations in
the presence of missing data to find relationships between and among
genotypes and phenotypes in mouse strains. The calculations are a
bottleneck for moderate to large problem sizes. Calculating PCC is
pervasive across bioinformatics, data analysis, phylogenetics,
statistics, stochastics, and anthropology. Our approach can be used
whenever a PCC matrix is computed.\\ \textbf{Results:} Our solution is
a reformulation of the PCC algorithm such that the lion's share of the
computation is done using matrix-matrix products and achieves 4.3
TFlop/s in single precision (77\% of the theoretical peak) on a single
Intel Xeon Gold 6148 CPU $@$ 2.4 GHz (Skylake) cpu. With R, for
example, this translates to as much as a 120x speedup.\\
\textbf{Availability:} Code is available for C\raise .8ex \hbox{$_{++}$} and The R Project for Statistical
Computing, under an GPL-v3 licence  at \href{https://github.com/UTennessee-JICS/MPCC}{https://github.com/UTennessee-JICS/MPCC}\\
\textbf{Contact:}
\href{glenn-brook@tennessee.edu}{glenn-brook@tennessee.edu}\\
\textbf{Supplementary information:} Supplementary data are
available at \textit{Bioinformatics} online.}

FIXME: download

\maketitle

\section{Introduction}
Biology has driven innovation in statistics from the early days. Pearson, Fischer,
Galton, and many other now famous statisticians were all working on biological
problems.
%The mathematical formula for Pearson$'$s correlation were derived by
%Auguste Bravais in 1844. However, as Stigler's Law \citep{Stigler1980} dictates,
%the name of the method credits Karl Pearson, who was building on ideas published
%by Francis Galton in the 1880s.

\enlargethispage{12pt}

Pearson$'$s correlation is one of the most used algorithms in science. Its use is
ubiquitous across all fields of science ranging from agriculture to zoology. Large
scale computation of correlations are found in many areas of biology and
bioinformatics.

Genotype correlations are used to construct haplotypes, build genetic maps, and
order markers within the genome. Furthermore, Pearson$'$s correlations are used in
co-expression analysis \citep{Tesson:2010}, (genome wide) association analysis,
reconstruction of genetic networks \citep{Fukushima:2013}, weighted correlation
network analysis (WGCNA) \citep{Horvath:2008} and correlated trait locus (CTL)
mapping \citep{Arends2016a}.

The BxD family is a panel of 150 recombinant inbred mice, and the BxD data collection
consists of 7000 hand curated classical phenotypes, high-density genotypes, and over
100 'omics' data sets. All BxD phenotype, genotype, and omics data is freely available on
\href{https://genenetwork.org/}{GeneNetwork.org} \citep{Sloan2016}. GeneNetwork is an
online platform for data storage and analysis, including Pearson$'$s correlation.

Computation of Pearson$'$s correlations in the R language for statistical computing
\citep{R:2005} is provided by the $cor()$ function. It is implemented in C++ and is
relatively performant when no missing data is present. In the presence of missing
data, two strategies are used to deal with the missing data.

Consider the case where, for a set of individuals, pairwise correlation is done
amongst the set of phenotypes for those individuals. Strategy 1 is to remove
all individuals that contain any missing phenotype data. This is undesirable in
biological data because often there are no individuals without at least some
missing data. Strategy 2 handles missing data on a case by case basis,
which significantly increases the algorithms runtime.

An additional issue with the $cor()$ function is the inability to, by default, use
multi-threading. This limitation can be solved by using the $parallel$ library.
However, not every end user has the ability to correctly setup the $parallel$
library, and the library uses a new R instance for each thread, adding significant
computational overhead.
\vspace*{-5mm}
\begin{figure}[!t]
  \centerline{\includegraphics[width=235pt]{img/figure01.eps}}
  \vspace*{-7mm}
  \caption{
    Genotype to genotype correlation between genetic markers on chromosome
    1 of the BxD family. Pairwise correlation between all 7321 marker was \textasciitilde{}6 times
    faster using MPCC.
  }
  \label{fig:fig1}
  \vspace*{-5mm}
\end{figure}
\begin{figure}[!t]
  \centerline{\includegraphics[width=235pt]{img/figure02.eps}}
  \vspace*{-7mm}
  \caption{
    Runtime speedup of MPCC versus the R $cor$ function shows
    increasing speedup with growing matrix sizes. However,
    speedup is negatively affected by increasing percentage of
    missing data, because of increased computational load in the
    missing data preparation step.
  }
  \label{fig:fig2}
  \vspace*{-5mm}
\end{figure}
\section{Approach}
\textbf{The Naive Version}\\
The Naive version of the MPCC algorithm was developed to be algorithmically identical
to the R $cor()$ function (Supplement 1 - 'The Naive Version'). Multi-threading
support without additional overhead is provided by OpenMP. The Naive version
provides a fair multi-threaded baseline against the Matrix version. It also provides
a multi-threaded fallback in R when no Intel\textregistered{} Math Kernel Library
(Intel\textregistered{} MKL) is available.\\
\textbf{The Matrix Version}\\
The Matrix version reformulates the naive version of the PCC algorithm
into a series of matrix-matrix products and element-wise matrix operations.
Missing data is handled by a bit-masking approach implemented in C.
Reformulation of the algorithm and our missing data strategy is described
in detail in Supplement 1 - 'The Matrix Version' and 'Missing Data Bit-masking'.
\vspace*{-5mm}
\begin{methods}
\section{Methods}
Two scenarios were designed to benchmark MPCC.\\
{\bf Scenario 1:} Pairwise correlation between genotypes of the BxD family.
Genotype data in the BxD family is almost complete, with only a few heterozygous
(missing) loci remaining, this scenario benchmarks MPCC versus $cor()$ when a limited
amount of missing data is present.\\
{\bf Scenario 2:} Speedup computation of MPCC versus $cor()$ in the presence of a
increasing amounts of missing data. Matrix sizes are increased in a step-wise fashion
to increase computational load.
\end{methods}
\vspace*{-2mm}
\section{Results}
Both the Naive and Matrix versions provide R interfaces designed as a drop-in
replacement for the $cor()$ function provided by R.

Using MPCC to compute genotype to genotype correlations (Fig \ref{fig:fig1})
showed a \textasciitilde{}5 times reduction in runtime compared to the
$cor()$ function provided in R.

With increasing matrix sizes  MPCC achieves an increasing speedup. However,
increasing amounts of missing data show a decreased speedup for MPCC, due
to missing Data Bit-masking. Multicore performance (40 threads) of MPCC
compared to the $cor()$ function, shows an 120 times reduction in runtime
when 5\% of data is missing, single core performance shows a 7.5 times
speedup (Fig \ref{fig:fig2}).

The Naive version using OpenMP to improve performance in comparison to the
$cor()$ function provides a consistent 3.5x speedup (56 core Intel Xeon E5-2680
v4 $@$ 2.40GHz - data not shown).

Using MPCC to compute genotype to genotype correlations (Fig \ref{fig:fig1}) leads to
an \textasciitilde{}6x speedup compared to the $cor()$ function provided in R.

Multicore performance (40 cores) of MPCC compared to the $cor()$ function, shows up to a
120 times reduction in runtime (Fig \ref{fig:fig2}). With increasing matrix sizes,
the MPCC algorithm achieves an increasing speedup due to the increased arithmetic
intensity of the algorithm. However, increasing the volume of missing data reduces the
efficiency of MPCC due to preprocessing steps.

\vspace*{-5mm}
\section{Discussion}
The Matrix version heavily leverages the Intel\textregistered{} MKL to obtain
maximum performance on Intel\textregistered{} systems. Performance of the
matrix version is due in part to the high arithmetic intensity of the
matrix-matrix product algorithm, and in part to the fact that the MKL
matrix-matrix product algorithm is highly optimized.

The missing data approach (Supplement 1 - 'Missing Data Bit-masking') can be
trivially generalized to other algorithms which rely on matrix-matrix multiplication.

Future work will investigate using OpenBLAS to provide a Free and Open-Source
Software (FOSS) alternative to the Intel\textregistered{} MKL, as well as the
feasibility of a GPU implementation.
\vspace*{-5mm}
\section{Conclusion}
Driven by demands in large data biology, computation of PCC was optimized.
MPCC greatly reduces the existing bottlenecks for many fields faced with
large problem sizes. The MPCC algorithm is much more efficient in its handling
of missing data, a common issue in bioinformatics, since missing data is
ubiquitous and leads to reduced performance.

A 120x speedup allows researchers to tackle many problems that could
not have been attempted before. MPCC can be used by any language which
allows calling C code. Furthermore, an R package with a drop-in
replacement for the $cor()$ function is provided.
\vspace*{-5mm}
\section*{Funding}
We thank the support of the UT Center for Integrative and Translational Genomics,
and funds from the UT-ORNL Governor's Chair, NIDA grant P30DA044223, NIAAA U01
AA013499 and U01 AA016662.
\vspace*{-5mm}
\bibliographystyle{natbib}
\bibliography{main}

\end{document}
